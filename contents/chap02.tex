% !TeX root = ../bjfuthesis-main.tex

\chapter{图表公式示例}
\section{公式}
公式的使用详见公式\eqref{equ:1}。
\begin{equation}
    R = \left\{ j | j \in  R \wedge \urcorner j \in S  \right\}
    \label{equ:1} %用于文内引用的标签
\end{equation}

\section{插图}

图片通常在 figure 环境中使用\cite{dupont1974bone}\cite{jianduju1994}  $\backslash$ includegraphics 插入,如图~\ref{fig:example} 的源代码。
建议矢量图片使用 PDF 格式,比如数据可视化的绘图;
照片应使用 JPG 格式;
其他的栅格图应使用无损的 PNG 格式。
注意,LaTeX 不支持 TIFF 格式;EPS 格式已经过时。

\begin{figure}[h]
	\centering
	\includegraphics[width=0.5\linewidth]{example-image-a.pdf}
	\caption*{此处添加图注,有图注或者其他说明时需要置于图题之上}
	\bicaption{示例图片标题} {English title}

	\label{fig:example}
\end{figure}

若图或表\ref{tab:three-line}中有附注,采用英文小写字母顺序编号,附注写在图或表的下方。
国外的期刊习惯将图表的标题和说明文字写成一段,需要改写为标题只含图表的名称,其他说明文字以注释方式写在图表下方,或者写在正文中。

如果一个图由两个或两个以上分图组成时,各分图分别以 (a)、(b)、(c)...... 作为图序,并须有分图题。
推荐使用 subcaption 宏包来处理, 比如图~\ref{fig:subfig-a} 和图~\ref{fig:subfig-b}。



\begin{figure}[h]
	\centering
	\subcaptionbox{分图 A\label{fig:subfig-a}}
	{\includegraphics[width=0.35\linewidth]{example-image-a.pdf}}
	\subcaptionbox{分图 B\label{fig:subfig-b}}
	{\includegraphics[width=0.35\linewidth]{example-image-b.pdf}}
	\bicaption{多个分图的示例} {English title}

	\label{fig:multi-image}
\end{figure}



\section{表格}

表应具有自明性。为使表格简洁易读,尽可能采用三线表,如表~\ref{tab:three-line}。
三条线可以使用 booktabs 宏包提供的命令生成。

\begin{table}[h]
	\xiaowu

	\centering
	\bicaption{三线表示例} {English title}
	\begin{tabular}{ll}
		\toprule
		文件名           & 描述                      \\
		\midrule
		bjfuthesis.cls   & 模板文件                  \\
		bjfuthesis-*.bst & BibTeX 参考文献表样式文件 \\
		\bottomrule
	\end{tabular}
	\label{tab:three-line}
\end{table}

表格如果有附注,尤其是需要在表格中进行标注时,可以使用 threeparttable宏包。

\begin{table}[h]
	\xiaowu
	\centering
	\begin{threeparttable}[c]
		\bicaption{带附注的表格示例} {English title}
		\label{tab:three-part-table}
		\begin{tabular}{ll}
			\toprule
			文件名                     & 描述                      \\
			\midrule
			bjfuthesis.cls\tnote{1}    & 模板文件                  \\
			bjfuthesis-*.bst \tnote{2} & BibTeX 参考文献表样式文件 \\
			\bottomrule
		\end{tabular}
		\begin{tablenotes}

			\item [1] 更新模板时,一定要记得编译生成 .cls 文件,否则编译论文时载入的依然是旧版的模板。
			\item [2] 更新模板时,一定要记得编译生成 .cls 文件,否则编译论文时载入的依然是旧版的模板。
		\end{tablenotes}
	\end{threeparttable}
\end{table}

如某个表需要转页接排,可以使用 longtable 宏包,需要在随后的各页上重复表的编号。
编号后跟表题(可省略)和“(续)”,置于表上方。续表均应重复表头。


\section{算法}

算法环境可以使用 algorithms 或者 algorithm2e 宏包。
\renewcommand{\algorithmicrequire}{\textbf{输入:}\unskip}
\renewcommand{\algorithmicensure}{\textbf{输出:}\unskip}

\begin{algorithm}[ht]
	\caption{Calculate $y = x^n$}
	\label{alg1}
	\small
	\begin{algorithmic}
		\REQUIRE $n \geq 0$
		\ENSURE $y = x^n$

		\STATE $y \leftarrow 1$, $X \leftarrow x$, $N \leftarrow n$

		\WHILE{$N \neq 0$}
		\IF{$N$ is even}
		\STATE $X \leftarrow X \times X$
		\STATE $N \leftarrow N / 2$
		\ELSE[$N$ is odd]
		\STATE $y \leftarrow y \times X$
		\STATE $N \leftarrow N - 1$
		\ENDIF
		\ENDWHILE
	\end{algorithmic}
\end{algorithm}
