% !TeX root = ../bjfuthesis-main.tex

\begin{acknowledgements}
我走了很远的路,吃了很多的苦,才将这份博士学位论文送到你的面前。二十二载求学路,一路风雨泥泞,许多不容易。如梦一场,仿佛昨天一家人才团聚过。\\
\par
出生在一个小山坳里,母亲在我十二岁时离家。父亲在家的日子不多,即便在我病得不能自己去医院的时候,也仅是留下勉強够治病的钱后又走了。我十七岁时,他因交通事故离世后,我哭得稀里糊涂,因为再得重病时没有谁来管我了。同年,和我住在一起的婆婆病放,真的无能为力。她照顾我十七年,下葬时却仅是一副薄薄的棺材。另一个家庭成员是老狗小花,为父亲和婆婆守过坟,后因我进城上高中而命不知何时何处所终。如兄长般的计算机启蒙老师邱浩没能看到我的大学录取通知书,对我照顾有加的师母也在不惑之前匆匆离开人世。每次回去看他们,这一座座坟茔都提示着生命的每一分钟都弥足珍贵。\\
\par
人情冷暖,生离死别,固然让人痛苦与无奈,而贫穷则可能让人失去希望。家徒四壁,在煤油灯下写作业或者读书都是晚上最开心的事。如果下雨,保留节目就是用竹笋壳塞瓦缝防漏雨。高中之前的主要经济来源是夜里抓黄鳞、周末钓鱼、养小猪崽和出租水牛,那些年里,方圆十公里的水田和小河都被我用脚测量过无数次。被狗和蛇追,半夜落水,因蓄电瓶进水而摸黑逃回家中;学费没交,黄鳝却被父亲偷卖了,然后买了肉和酒,都是难以避免的事。\\
\par
人后的苦尚且还能克服,人前的尊严却无比脆弱。上课的时候,因拖欠学费而经常被老师叫出教室约谈。雨天湿漉着上课,屁股后面说不定还是泥。夏天光着脚走在滚烫的路上。冬天穿着破旧衣服打着寒颤穿过那条长长的过道领作业本。这些都可能成为压垮骆驼的最后一根稻草。如果不是考试后常能从主席台领奖金,顺便能贴一墙奖状满足最后的虚荣心,我可能早已放弃。\\
\par
身处命运的漩涡,耗尽心力去争取那些可能本就是稀松平常的东西,每次转折都显得那么的身不由己。幸运的是,命运到底还有一丝怜惜。进入高中后,学校免了全部学杂费,胡叔叔一家帮助解决了生活费。进入大学后,计算机终于成了我一生的事业与希望,胃溃疡和胃出血也终与我作别。\\
\par
从家出发坐大巴需要两个半小时才能到县城,一直盼着走出大山。从矩光乡小学、大寅镇中学、仪陇县中学、绵阳市南山中学,到重庆的西南大学,再到中科院自动化所,我也记不清有多少次因为现实的压力而觉得自己快扛不下去了。这一路,信念很简单,把书念下去,然后走出去,不枉活一世。世事难料,未来注定还会面对更为复杂的局面。但因为有了这些点点滴滴,我已经有勇气和耐心面对任何困难和挑战。理想不伟大,只愿年过半百,归来仍是少年,希望还有机会重新认识这个世界,不辜负这一生吃过的苦。最后如果还能做出点让别人生活更美好的事,那这辈子就赚了。
\end{acknowledgements}
